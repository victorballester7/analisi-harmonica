\documentclass[10pt,a4paper]{article}
\usepackage[utf8]{inputenc}
\usepackage{amsthm, amsmath, mathtools, amssymb}
\usepackage[intlimits]{esint}
\usepackage[left=1.5cm,right=1.5cm,top=2cm,bottom=2cm]{geometry}
\usepackage[colorlinks,linkcolor=blue,citecolor=blue,urlcolor=blue]{hyperref}
\usepackage[catalan]{babel}
\usepackage{titlesec}
\usepackage{enumitem}
\usepackage{physics}
\usepackage{fancyhdr}

\newcommand{\NN}{\ensuremath{\mathbb{N}}} % set of natural numbers
\newcommand{\ZZ}{\ensuremath{\mathbb{Z}}} % set of integers
\newcommand{\QQ}{\ensuremath{\mathbb{Q}}} % set of rationals
\newcommand{\RR}{\ensuremath{\mathbb{R}}} % set of real numbers
\newcommand{\CC}{\ensuremath{\mathbb{C}}} % set of complex numbers
\newcommand{\KK}{\ensuremath{\mathbb{K}}} % a general field

\newcommand{\vf}[1]{\boldsymbol{\mathrm{#1}}} % math style for vectors and matrices and vector-values functions (previously it was \*vb{#1} but this does not apply to greek letters)
\newcommand{\ii}{\mathrm{i}} % imaginary unit
\renewcommand{\O}{\mathrm{O}} % big O-notation
\DeclareMathOperator{\supp}{\mathrm{supp}} % support

\newtheorem{theorem}{Teorema}
\newtheorem{exercici}{Exercici}
\newtheorem{prop}{Proposició}
\theoremstyle{definition}
\newtheorem{definition}{Definició}
\theoremstyle{remark}
\newtheorem*{res}{Resolució}
\DeclareDocumentCommand\derivative{ s o m g d() }{ 
  % Total derivative
  % s: star for \flatfrac flat derivative
  % o: optional n for nth derivative
  % m: mandatory (x in df/dx)
  % g: optional (f in df/dx)
  % d: long-form d/dx(...)
    \IfBooleanTF{#1}
    {\let\fractype\flatfrac}
    {\let\fractype\frac}
    \IfNoValueTF{#4}
    {
        \IfNoValueTF{#5}
        {\fractype{\diffd \IfNoValueTF{#2}{}{^{#2}}}{\diffd #3\IfNoValueTF{#2}{}{^{#2}}}}
        {\fractype{\diffd \IfNoValueTF{#2}{}{^{#2}}}{\diffd #3\IfNoValueTF{#2}{}{^{#2}}} \argopen(#5\argclose)}
    }
    {\fractype{\diffd \IfNoValueTF{#2}{}{^{#2}} #3}{\diffd #4\IfNoValueTF{#2}{}{^{#2}}}\IfValueT{#5}{(#5)}}
} % differential operator
\DeclareDocumentCommand\partialderivative{ s o m g d() }{ 
  % Total derivative
  % s: star for \flatfrac flat derivative
  % o: optional n for nth derivative
  % m: mandatory (x in df/dx)
  % g: optional (f in df/dx)
  % d: long-form d/dx(...)
  \IfBooleanTF{#1}
    {\let\fractype\flatfrac}
    {\let\fractype\frac}
    \IfNoValueTF{#4}{
      \IfNoValueTF{#5}
      {\fractype{\partial \IfNoValueTF{#2}{}{^{#2}}}{\partial #3\IfNoValueTF{#2}{}{^{#2}}}}
      {\fractype{\partial \IfNoValueTF{#2}{}{^{#2}}}{\partial #3\IfNoValueTF{#2}{}{^{#2}}} \argopen(#5\argclose)}
    }
    {\fractype{\partial \IfNoValueTF{#2}{}{^{#2}} #3}{\partial #4\IfNoValueTF{#2}{}{^{#2}}}\IfValueT{#5}{(#5)}}
} % partial differential operator

\titleformat{\section}
  {\normalfont\fontsize{11}{15}\bfseries}{\thesection}{1em}{}


%%% HARMONIC ANALYSIS
\newcommand{\F}{\mathcal{F}} % Fourier transform operator
\renewcommand{\pv}{\mathrm{p.v.}} % Cauchy principal value

% \renewcommand{\theenumi}{\textbf{\arabic{enumi}}}
\renewcommand{\theenumi}{\alph{enumi}}
\renewcommand{\theenumiii}{\roman{enumiii}}

\renewcommand{\exp}[1]{\mathrm{e}^{#1}} % exponential function
\DeclareMathOperator*{\im}{Im}
\DeclareMathOperator*{\sgn}{sgn}

\setlength\multlinegap{0pt} % disable the margins on \begin{multline} command.
  \allowdisplaybreaks


\title{\bfseries\Large Llista de problemes 6}

\author{Víctor Ballester Ribó\\NIU: 1570866}
\date{\parbox{\linewidth}{\centering
  Anàlisi Harmònica\endgraf
  Grau en Matemàtiques\endgraf
  Universitat Autònoma de Barcelona\endgraf
  Maig de 2023}}
  \pagestyle{fancy}
  \fancyhf{}
  \fancyhfoffset[L]{1cm}
  \fancyhfoffset[R]{1cm}
  \rhead{NIU: 1570866}
  \lhead{Víctor Ballester}
  \cfoot{\thepage}
  %\setlength{\headheight}{13.6pt}

\setlength{\parindent}{0pt}
\begin{document}
\selectlanguage{catalan}
\maketitle
\setcounter{exercici}{1}
\begin{exercici}
  Sigui $p(x)$ un polinomi. Llavors existeix una distribució $T\in\mathcal{D}^*(\RR)$ tal que $p(x)T=1$.
\end{exercici}
\begin{res}
  Recordem d'un seminari anterior les distribucions $T_m=\frac{{(-1)}^{m-1}}{(m-1)!}\partial^m\log\abs{x}\in \mathcal{D}^*(\RR)$ que satisfan $x^mT_m=1$, $m\in\NN$. Fent una translació tenim que les distribucions $S_{m,\alpha}:=\frac{{(-1)}^{m-1}}{(m-1)!}\partial^m\log\abs{x-\alpha}$ satisfan ${(x-\alpha)}^mS_m=1$ per a tot $\alpha\in\RR$.
  Aleshores, fent la descomposició per fraccions simples de $p(x)$ tenim que
  $$
    \frac{1}{p(x)} = \sum_{i=0}^n \frac{c_i}{{(x-\alpha_i)}^{n_i}}+\sum_{i=0}^{m} \frac{q_i(x)}{r_i(x)^{m_i}}
  $$
  per a certs polinomis $q_i$ i certes constants $c_i$. Els polinomis $r_i$ representen els factors irreductibles de $p$, i per tant, com que no s'anu\lgem en $\RR$, $\frac{1}{{r_i(x)}^{m_i}}\in L_\mathrm{loc}^1(\RR)$. Ara definim $T:= \sum_{i=0}^n c_iS_{n_i,\alpha_i} +\sum_{i=0}^{m} \frac{q_i(x)}{r_i(x)^{m_i}}$, que és una distribució ja que és una combinació lineal de distribucions. Tenim que:
  $$
    p(x)T=p(x)\left(\sum_{i=0}^n c_iS_{n_i,\alpha_i}+\sum_{i=0}^{m} \frac{q_i(x)}{r_i(x)^{m_i}}\right) =p(x) \left(\sum_{i=0}^n \frac{c_i}{{(x-\alpha_i)}^{n_i}}+\sum_{i=0}^{m} \frac{q_i(x)}{r_i(x)^{m_i}}\right) =1
  $$
\end{res}
\begin{exercici}
  Sigui $T\in \mathcal{D}^*(\RR)$. El \emph{suport} de $T$ es defineix com la intersecció de tots els tancats $K$ amb la propietat següent: si $\varphi\in\mathcal{D}(\RR)$ té suport a $\RR^n \setminus K$, aleshores $\langle T,\varphi\rangle=0$.

  Considerem l'espai $\mathcal{E}(\RR^n):=\mathcal{C}^\infty(\RR^n)$ i el seu dual $\mathcal{E}^*(\RR^n)$. És fàcil veure que $T\in \mathcal{E}^*(\RR^n)$ si i només si existeixen $C>0$ i $N, m\in\NN$ tals que
  \begin{equation}\label{eq:1}
    \abs{\langle T,\varphi\rangle}\leq C\sum_{\abs{\alpha}\leq m}\sup_{\abs{x}\leq N}\abs{(\partial^\alpha\varphi)(x)}\quad\text{per a tota }\varphi\in\mathcal{E}(\RR^n).
  \end{equation}
  \begin{enumerate}
    \item Demostreu que $T$ és una distribució amb suport compacte si i només si $T\in \mathcal{E}^*(\RR^n)$.
    \item Si $T\in \mathcal{S}^*(\RR^n)$ està suportada en un punt $x_0$, llavors:
          $$
            T=\sum_{\abs{\alpha}\leq k}a_\alpha\partial^\alpha\delta_{x_0}
          $$
    \item Deduïu que si $T\in \mathcal{S}^*(\RR^n)$ és tal que $\widehat{T}$ està suportada en $\xi_0$, llavors $T$ és combinació lineal finita de funcions ${(-2\pi\ii\xi)}^\alpha\exp{2\pi i\langle \xi,\xi_0\rangle}$. En particular, si $\widehat{T}$ està suportada a l'origen, llavors $T$ és un polinomi.
    \item Deduïu que si $u\in \mathcal{S}^*(\RR^n)$ és tal que $\Delta u=0$, llavors $u$ és un polinomi.
  \end{enumerate}
\end{exercici}
\begin{res}\hfill
  \begin{enumerate}
    \item Suposem primer que $T$ és una distribució amb suport compacte $K$. Per definició si $\varphi\in \mathcal{E}(\RR^n)$ té suport a $\RR^n\setminus K$, tenim que $\langle T,\varphi\rangle =0$. Ara considerem una funció $\psi\in \mathcal{D}(\RR^n)$ tal que $\psi(x)=1$ per a tot $x\in K$. Aleshores, per a tota $\varphi\in \mathcal{E}(\RR^n)$ tenim que:
          $$
            \abs{\langle T,\varphi\rangle} = \abs{\langle T,\varphi\psi\rangle + \langle T,\varphi(1-\psi)\rangle} = \abs{\langle T,\varphi\psi\rangle}\leq C \sum_{\abs{\alpha}\leq m}\sup_{x\in Q}\abs{(\partial^\alpha(\varphi\psi))(x)}
          $$
          per a tot compacte $Q\subseteq \Omega$ i certes constants $C>0$, $m\in\NN$. La segona igualtat es deu al fet que $\varphi(1-\psi)$ té suport contingut en $\RR^n\setminus K$ i la desigualtat bé del fet que $\varphi\psi\in\mathcal{D}(\RR^n)$ i, per tant, sobre aquestes funcions tenim continuïtat. Si triem $Q=K$ tenim que $\sup_{x\in Q}\abs{(\partial^\alpha(\varphi\psi))(x)}= \sup_{x\in K}\abs{(\partial^\alpha\varphi)(x)}$ ja que $\varphi\psi=\varphi$ a $K$. Finalment, triant un $N\in\NN$ suficientment gran perquè $K\subseteq \overline{B(0,N)}$ obtenim el resultat: $$\abs{\langle T,\varphi\rangle}\leq C\sum_{\abs{\alpha}\leq m} \sup_{x\in K}\abs{(\partial^\alpha\varphi)(x)}\leq C\sum_{\abs{\alpha}\leq m}\sup_{\abs{x}\leq N}\abs{(\partial^\alpha\varphi)(x)}$$


          Ara suposem que $T\in \mathcal{E}^*(\RR^n)$. Com que $\mathcal{D}(\RR^n)\subset \mathcal{E}(\RR^n)$, tenim que $\mathcal{E}^*(\RR^n)\subset \mathcal{D}^*(\RR^n)$. Per tant, $T$ és una distribució i a més el seu suport $K$ és tancat, perquè és intersecció de tancats. Cal veure que $K$ és acotat. Del fet que $T\in\mathcal{E}^*(\RR^n)$, tenim que existeixen constants $C>0$, $N,m\in\NN$ tals que es satisfà \eqref{eq:1}. Ara bé, si $K$ no fos acotat, aleshores podríem triar $\varphi\in \mathcal{E}(\RR^n)$ tal que $\supp \varphi \subseteq \RR^n\setminus \overline{B(0, N)}$ i $\abs{\langle T,\varphi\rangle}>0$, ja que si no poguéssim voldria dir que $\supp T\subseteq \overline{B(0, N)}$, que estaria acotat. Però això contradiu \eqref{eq:1} ja que totes les seminormes $\sup_{\abs{x}\leq N}\abs{(\partial^\alpha\varphi)(x)}$ serien 0 perquè $\varphi$ és nu\lgem a dins de $\overline{B(0, N)}$. Amb això hem vist que no només $K$ ha de ser acotat, sinó que ha d'estar contingut en $\overline{B(0,N)}$.

    \item Sigui $\varepsilon>0$, $\varphi\in \mathcal{S}(\RR^n)$ i $\omega\in \mathcal{D}(\RR^n)$ una \emph{bump function} que val 1 en $B(0,1)$ i 0 en $\RR^n\setminus B(0,2)$.
          Expandint $\varphi$ en sèrie de Taylor centrada a $x_0$ tenim que si $k\in \NN$ (pendent a determinar), aleshores:
          $$
            \varphi(x) = \sum_{\abs{\alpha}\leq k}\frac{1}{\alpha!}\partial^\alpha\varphi(x_0){(x-x_0)}^\alpha + R_{k,\varepsilon}(x)
          $$
          on $\displaystyle R_{k,\varepsilon}(x) = \sum_{\abs{\alpha}=k+1}\frac{\partial^\alpha\varphi(x_0+c(x-x_0))}{\alpha!}{(x-x_0)}^\alpha$,
          amb $c\in(0,1)$ i $\abs{x-x_0}\leq\varepsilon$, és el residu de Taylor. Fixem-nos que si denotem $a_\alpha:=\displaystyle\frac{{(-1)}^\alpha}{\alpha!}\langle T,{(x-x_0)}^\alpha\rangle$, aleshores:
          $$
            \langle T,\varphi\rangle = \sum_{\abs{\alpha}\leq k}a_\alpha{(-1)}^\alpha\partial^\alpha\varphi(x_0) + \langle T,R_{k,\varepsilon}\rangle=\sum_{\abs{\alpha}\leq k}a_\alpha \langle \partial^\alpha\delta_{x_0},\varphi\rangle + \langle T,R_{k,\varepsilon}\rangle
          $$
          Hem de veure que $\abs{\langle T,R_{k,\varepsilon}\rangle}\overset{\varepsilon\to 0}{\longrightarrow}0$. Definim $\omega_\varepsilon(x):=\omega((x-x_0)/\varepsilon)$, que val 1 en $B(x_0,\varepsilon)$ i 0 en $\RR^n\setminus B(x_0,2\varepsilon)$. Clarament $R_{k,\varepsilon}\in \mathcal{S}(\RR^n)$, i per tant, $R_{k,\varepsilon}\omega_\varepsilon\in \mathcal{S}(\RR^n)$. Per tant:
          $$
            \langle T,R_{k,\varepsilon} \rangle = \langle T,R_{k,\varepsilon}\omega_\varepsilon\rangle + \langle T,R_{k,\varepsilon}(1-\omega_\varepsilon)\rangle = \langle T,R_{k,\varepsilon}\omega_\varepsilon\rangle
          $$
          ja que $R_{k,\varepsilon}(1-\omega_\varepsilon)$ té suport a $\RR^n\setminus B(x_0,\varepsilon)\subset \RR^n\setminus\{x_0\}$. Usant l'apartat anterior, tenim que $T\in \mathcal{E}^*(\RR^n)$ i, per tant, $\exists C\in\RR$ i $N,m\in\NN$ tals que:
          $$
            \abs{\langle T,R_{k,\varepsilon}\omega_\varepsilon\rangle}\leq C\sum_{\abs{\beta}\leq m}\sup_{\abs{x}\leq N}\abs{\partial^\beta (R_{k,\varepsilon}\omega_\varepsilon)(x)} = C\sum_{\abs{\beta}\leq m}\sup_{\substack{\abs{x}\leq N\\ \abs{x-x_0}\leq 2\varepsilon}}\abs{\partial^\beta (R_{k,\varepsilon}\omega_\varepsilon)(x)}
          $$

          Fent servir la regla de Leibniz per $\beta\in (\NN\cup\{0\})^d$ i usant que $\abs{x-x_0}\leq 2 \varepsilon$ tenim que:
          $$
            \abs{\partial^\beta (R_{k,\varepsilon}\omega_\varepsilon)(x)} = \abs{\sum_{\gamma\leq \beta}\binom{\beta}{\gamma}\partial^\gamma \omega_\varepsilon\partial^{\beta-\gamma} R_{k,\varepsilon}}\lesssim\sum_{\gamma\leq \beta} \frac{1}{\varepsilon^{\abs{\gamma}}}\abs{\varepsilon}^{k+1 - (\abs{\beta}-\abs{\gamma})}\lesssim \varepsilon^{k+1-\abs{\beta}}
          $$
          on hem usat que $\abs{\partial^{\beta-\gamma} {(x-x_0)}^\alpha}\simeq \abs{x-x_0}^{\abs{\alpha} - (\abs{\beta}-\abs{\gamma})}$ i $\abs{\alpha} = k+1$ i, per tant, $\abs{\partial^{\beta-\gamma}R_{k,\varepsilon}}\lesssim \abs{\varepsilon}^{k+1 - (\abs{\beta}-\abs{\gamma})}$ (també fent Leibniz). Per tant, com que $\abs{\beta} \leq m$, triant $k=m$ tenim que:
          $$
            \abs{\langle T,R_{k,\varepsilon}\omega_\varepsilon\rangle}\leq C\sum_{\abs{\beta}\leq m}\sup_{\substack{\abs{x}\leq N\\ \abs{x-x_0}\leq 2\varepsilon}}\abs{\partial^\beta (R_{k,\varepsilon}\omega_\varepsilon)(x)}\lesssim\sum_{\abs{\beta}\leq m}\varepsilon^{m+1-\abs{\beta}}\overset{\varepsilon\to 0}{\longrightarrow}0
          $$
    \item De l'apartat anterior tenim que $\displaystyle\widehat{T} = \sum_{\abs{\alpha}\leq k}c_\alpha\partial^\alpha\delta_{\xi_0}$. Prenent $\F^{-1}$ a ambdós costats i usant que
          $$
            \F^{-1}({\partial^\alpha\delta_{\xi_0}})= \F^3({\partial^\alpha\delta_{-\xi_0}}) = \F^2({(2\pi\ii\xi)}^\alpha\widehat{\delta_{\xi_0}})= {(-2\pi\ii\xi)}^\alpha\widehat{\delta_{-\xi_0}}={(-2\pi\ii\xi)}^\alpha \exp{2\pi i \xi\cdot\xi_0}
          $$
          obtenim:
          $$
            T = \sum_{\abs{\alpha}\leq k}c_\alpha\F^{-1}({\partial^\alpha\delta_{\xi_0}}) =\sum_{\abs{\alpha}\leq k}c_\alpha{(-2\pi\ii\xi)}^\alpha\exp{2\pi i \xi\cdot\xi_0}
          $$
          En particular si $\xi_0=0$, aleshores ${T} = \sum_{\abs{\alpha}\leq k}c_\alpha{(-2\pi\ii\xi)}^\alpha$, que és un polinomi en $\xi$.
    \item Fent transformada de Fourier a l'equació obtenim:
          $$
            0=\widehat{\Delta u} =\sum_{i=0}^n \widehat{{\partial_i}^2 u} = \sum_{i=0}^n {(2\pi\ii\xi_i)}^2\widehat{u} = -4\pi^2\abs{\xi}^2\widehat{u}
          $$
          Ara sigui $\varepsilon>0$ i $B_\varepsilon:=\overline{B(0,\varepsilon)}$. Considerem $\varphi\in\mathcal{S}(\RR^n)$ tal que $\supp\varphi\subseteq \RR^n\setminus B_\varepsilon$. Aleshores, la funció $\phi(\xi):=\frac{\varphi(\xi)}{\abs{\xi}^2}$ està ben definida is pertant a $\mathcal{S}(\RR^n)$. En efecte, si $\xi\in B_{\varepsilon/2}$, com que totes les derivades de $\varphi$ s'anu\lgem en en $B_{\varepsilon/2}$, tenim que:
          $$\abs{\phi(\xi)} = \abs{\frac{\varphi(\xi)-\varphi(0)-\grad \varphi(0)\cdot \xi}{\abs{\xi}^2}}\leq \abs{H\varphi(\eta)}\frac{\abs{\xi}^2}{2\abs{\xi}^2} =0$$
          ja que $\abs{\eta} \leq\abs{\xi}$. Per tant, $\phi\in\mathcal{S}(\RR^n)$ i llavors:
          $$
            \langle  u,\varphi\rangle = \langle u,\abs{\xi}^2\phi\rangle = \langle \abs{\xi}^2u,\phi\rangle = 0
          $$
          Com que això és vàlid $\forall \varepsilon>0$, tenim que $\widehat{u}$ té suport contingut a $\{0\}$ (i no pot ser $\varnothing$, perquè si no seria la solució trivial), i per tant, $u$ és un polinomi, per l'apartat anterior.
  \end{enumerate}
\end{res}
\end{document}
