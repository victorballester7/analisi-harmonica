\documentclass[10pt,a4paper]{article}
\usepackage[utf8]{inputenc}
\usepackage{amsthm, amsmath, mathtools, amssymb}
\usepackage[intlimits]{esint}
\usepackage[left=1.5cm,right=1.5cm,top=2cm,bottom=2cm]{geometry}
\usepackage[colorlinks,linkcolor=blue,citecolor=blue,urlcolor=blue]{hyperref}
\usepackage[catalan]{babel}
\usepackage{titlesec}
\usepackage{enumitem}
\usepackage{physics}
\usepackage{fancyhdr}

\newcommand{\NN}{\ensuremath{\mathbb{N}}} % set of natural numbers
\newcommand{\ZZ}{\ensuremath{\mathbb{Z}}} % set of integers
\newcommand{\QQ}{\ensuremath{\mathbb{Q}}} % set of rationals
\newcommand{\RR}{\ensuremath{\mathbb{R}}} % set of real numbers
\newcommand{\CC}{\ensuremath{\mathbb{C}}} % set of complex numbers
\newcommand{\KK}{\ensuremath{\mathbb{K}}} % a general field

\newcommand{\vf}[1]{\boldsymbol{\mathrm{#1}}} % math style for vectors and matrices and vector-values functions (previously it was \*vb{#1} but this does not apply to greek letters)
\newcommand{\ii}{\mathrm{i}} % imaginary unit
\renewcommand{\O}{\mathrm{O}} % big O-notation
\DeclareMathOperator{\supp}{\mathrm{supp}} % support

\newtheorem{theorem}{Teorema}
\newtheorem{exercici}{Exercici}
\newtheorem{prop}{Proposició}
\theoremstyle{definition}
\newtheorem{definition}{Definició}
\theoremstyle{remark}
\newtheorem*{res}{Resolució}
\DeclareDocumentCommand\derivative{ s o m g d() }{ 
  % Total derivative
  % s: star for \flatfrac flat derivative
  % o: optional n for nth derivative
  % m: mandatory (x in df/dx)
  % g: optional (f in df/dx)
  % d: long-form d/dx(...)
    \IfBooleanTF{#1}
    {\let\fractype\flatfrac}
    {\let\fractype\frac}
    \IfNoValueTF{#4}
    {
        \IfNoValueTF{#5}
        {\fractype{\diffd \IfNoValueTF{#2}{}{^{#2}}}{\diffd #3\IfNoValueTF{#2}{}{^{#2}}}}
        {\fractype{\diffd \IfNoValueTF{#2}{}{^{#2}}}{\diffd #3\IfNoValueTF{#2}{}{^{#2}}} \argopen(#5\argclose)}
    }
    {\fractype{\diffd \IfNoValueTF{#2}{}{^{#2}} #3}{\diffd #4\IfNoValueTF{#2}{}{^{#2}}}\IfValueT{#5}{(#5)}}
} % differential operator
\DeclareDocumentCommand\partialderivative{ s o m g d() }{ 
  % Total derivative
  % s: star for \flatfrac flat derivative
  % o: optional n for nth derivative
  % m: mandatory (x in df/dx)
  % g: optional (f in df/dx)
  % d: long-form d/dx(...)
  \IfBooleanTF{#1}
    {\let\fractype\flatfrac}
    {\let\fractype\frac}
    \IfNoValueTF{#4}{
      \IfNoValueTF{#5}
      {\fractype{\partial \IfNoValueTF{#2}{}{^{#2}}}{\partial #3\IfNoValueTF{#2}{}{^{#2}}}}
      {\fractype{\partial \IfNoValueTF{#2}{}{^{#2}}}{\partial #3\IfNoValueTF{#2}{}{^{#2}}} \argopen(#5\argclose)}
    }
    {\fractype{\partial \IfNoValueTF{#2}{}{^{#2}} #3}{\partial #4\IfNoValueTF{#2}{}{^{#2}}}\IfValueT{#5}{(#5)}}
} % partial differential operator

\titleformat{\section}
  {\normalfont\fontsize{11}{15}\bfseries}{\thesection}{1em}{}


%%% HARMONIC ANALYSIS
\newcommand{\F}{\mathcal{F}} % Fourier transform operator
\renewcommand{\pv}{\mathrm{p.v.}} % Cauchy principal value

% \renewcommand{\theenumi}{\textbf{\arabic{enumi}}}
\renewcommand{\theenumi}{\alph{enumi}}
\renewcommand{\theenumiii}{\roman{enumiii}}

\renewcommand{\exp}[1]{\mathrm{e}^{#1}} % exponential function
\DeclareMathOperator*{\im}{Im}
\DeclareMathOperator*{\sgn}{sgn}

\setlength\multlinegap{0pt} % disable the margins on \begin{multline} command.
  \allowdisplaybreaks


\title{\bfseries\Large Llista de problemes 6}

\author{Víctor Ballester Ribó\\NIU: 1570866}
\date{\parbox{\linewidth}{\centering
  Anàlisi Harmònica\endgraf
  Grau en Matemàtiques\endgraf
  Universitat Autònoma de Barcelona\endgraf
  Maig de 2023}}
  \pagestyle{fancy}
  \fancyhf{}
  \fancyhfoffset[L]{1cm}
  \fancyhfoffset[R]{1cm}
  \rhead{NIU: 1570866}
  \lhead{Víctor Ballester}
  \cfoot{\thepage}
  %\setlength{\headheight}{13.6pt}

\setlength{\parindent}{0pt}
\begin{document}
\selectlanguage{catalan}
\maketitle
\setcounter{exercici}{1}
\begin{exercici}
  Sigui $p(x)$ un polinomi. Llavors existeix una distribució $T\in\mathcal{D}^*(\RR)$ tal que $p(x)T=1$.
\end{exercici}
\begin{res}
  Recordem d'un seminari anterior les distribucions $T_m=\frac{{(-1)}^{m-1}}{(m-1)!}\partial^m\log\abs{x}\in \mathcal{D}^*(\RR)$ que satisfan $x^mT_m=1$, $m\in\NN$. Fent una translació tenim que les distribucions $S_{m,\alpha}:=\frac{{(-1)}^{m-1}}{(m-1)!}\partial^m\log\abs{x-\alpha}$ satisfan ${(x-\alpha)}^mS_m=1$ per a tot $\alpha\in\RR$.
  Aleshores, fent la descomposició per fraccions simples de $p(x)$ tenim que
  $$
    \frac{1}{p(x)} = \sum_{i=0}^n \frac{q_i(x)}{{(x-\alpha_i)}^{n_i}}
  $$
  per a certs polinomis $q_i$ tals que $\deg q_i<n_i$. Ara definim $T:= \sum_{i=0}^n q_i(x)S_{n_i,\alpha_i}$. Aleshores tenim que:
  $$
    p(x)T=p(x)\sum_{i=0}^n q_i(x)S_{n_i,\alpha_i} =p(x) \sum_{i=0}^n \frac{q_i(x)}{{(x-\alpha_i)}^{n_i}} =1
  $$
\end{res}
\begin{exercici}
  Sigui $T\in \mathcal{D}^*(\RR)$. El \emph{suport} de $T$ es defineix com la intersecció de tots els tancats $K$ amb la propietat següent: si $\varphi\in\mathcal{D}(\RR)$ té suport a $\RR^n \setminus K$, aleshores $\langle T,\varphi\rangle=0$.

  Considerem l'espai $\mathcal{E}(\RR^n):=\mathcal{C}^\infty(\RR^n)$ i el seu dual $\mathcal{E}^*(\RR^n)$. És fàcil veure que $T\in \mathcal{E}^*(\RR^n)$ si i només si existeixen $C>0$ i $N, m\in\NN$ tals que
  $$
    \abs{\langle T,\varphi\rangle}\leq C\sum_{\abs{\alpha}\leq m}\sup_{\abs{x}\leq N}\abs{(\partial^\alpha\varphi)(x)}\quad\text{per a tota }\varphi\in\mathcal{E}(\RR^n).
  $$
  \begin{enumerate}
    \item Demostreu que $T$ és una distribució amb suport compacte si i només si $T\in \mathcal{E}^*(\RR^n)$.
    \item Si $T\in \mathcal{S}^*(\RR^n)$ està suportada en un punt $x_0$, llavors:
          $$
            T=\sum_{\abs{\alpha}\leq k}a_\alpha\partial^\alpha\delta_{x_0}
          $$
    \item Deduïu que si $T\in \mathcal{S}^*(\RR^n)$ és tal que $\widehat{T}$ està suportada en $\xi_0$, llavors $T$ és combinació lineal finita de funcions ${(-2\pi\ii\xi)}^\alpha\exp{2\pi i\langle \xi,\xi_0\rangle}$. En particular, si $\widehat{T}$ està suportada a l'origen, llavors $T$ és un polinomi.
    \item Deduïu que si $u\in \mathcal{S}^*(\RR^n)$ és tal que $\Delta u=0$, llavors $u$ és un polinomi.
  \end{enumerate}
\end{exercici}
\begin{res}\hfill
  \begin{enumerate}
    \item Suposem primer que $T$ és una distribució amb suport compacte $K$. Per definició si $\varphi\in \mathcal{E}(\RR^n)$ té suport a $\RR^n\setminus K$, tenim que $\langle T,\varphi\rangle =0$. Ara per a una $\varphi\in \mathcal{E}(\RR^n)$ general, considerem una funció $\psi\in \mathcal{D}(\RR^n)$ tal que $\psi(x)=1$ per a tot $x\in K$. Aleshores, per a tota $\varphi\in \mathcal{E}(\RR^n)$ tenim que:
          $$
            \abs{\langle T,\varphi\rangle} = \abs{\langle T,\varphi\psi\rangle + \langle T,\varphi(1-\psi)\rangle} = \abs{\langle T,\varphi\psi\rangle}\leq C \sum_{\abs{\alpha}\leq m}\sup_{\abs{x}\leq N}\abs{(\partial^\alpha(\varphi\psi))(x)}
          $$
          per certes $C>0$, $m,N\in\NN$. La segona igualtat es deu al fet que $\varphi(1-\psi)$ té support contingut en $\RR^n\setminus K$ i l'última desigualtat bé del fet que $\varphi\psi\in\mathcal{D}(\RR^n)$ i, per tant, sobre aquestes funcions tenim continuïtat.

          Ara suposem que $T\in \mathcal{E}^*(\RR^n)$. Com que $\mathcal{D}(\RR^n)\subset \mathcal{E}(\RR^n)$, tenim que $\mathcal{E}^*(\RR^n)\subset \mathcal{D}^*(\RR^n)$. Per tant, $T$ és una distribució i a més el seu suport $K$ és tancat, perquè és intersecció de tancats. Cal veure que $K$ és acotat. Si no ho fos, aleshores podríem trobar una successió $(x_n)\in K$ tal que $x_n\to \infty$.  Per a construir aquesta $\varphi$, considerem una $(\phi_n)\in\mathcal{S}(\RR^n)$

          aleshores podríem trobar una $\varphi\in\mathcal{E}(\RR^n)$ que tendís a $\infty$ dins de $K$, i per tant, tindríem que $T$ no podria estar en $\mathcal{E}^*(\RR^n)$, que és una contradicció.
    \item Tenim que:
          $$
            \langle T,\varphi\rangle = \langle T, \varphi \vf{1}_{\{x_0\}}+(1-  \vf{1}_{\{x_0\}})\varphi\rangle = \langle T, \varphi \vf{1}_{\{x_0\}}\rangle
          $$
    \item De l'apartat anterior tenim que:
          $$
            \widehat{T} = \sum_{\abs{\alpha}\leq k}c_\alpha\widehat{\partial^\alpha\delta_{\xi_0}} = \sum_{\abs{\alpha}\leq k}c_\alpha{(2\pi\ii\xi)}^\alpha\widehat{\delta_{\xi_0}}= \sum_{\abs{\alpha}\leq k}c_\alpha{(2\pi\ii\xi)}^\alpha\exp{-2\pi i \xi\cdot\xi_0}
          $$
          En particular si $\xi_0=0$, aleshores $\widehat{T} = \sum_{\abs{\alpha}\leq k}c_\alpha{(2\pi\ii\xi)}^\alpha$, que és un polinomi en $\xi$.
    \item Fent transformada de Fourier a l'equació obtenim:
          $$
            0=\widehat{\Delta u} =\sum_{i=0}^n \widehat{{\partial_i}^2 u} = \sum_{i=0}^n {(2\pi\ii\xi_i)}^2\widehat{u} = -4\pi^2\abs{\xi}^2\widehat{u}
          $$
  \end{enumerate}
\end{res}
\end{document}
