\documentclass[10pt,a4paper]{article}
\usepackage[utf8]{inputenc}
\usepackage{amsthm, amsmath, mathtools, amssymb}
\usepackage[intlimits]{esint}
\usepackage[left=1.5cm,right=1.5cm,top=2cm,bottom=2cm]{geometry}
\usepackage[colorlinks,linkcolor=blue,citecolor=blue,urlcolor=blue]{hyperref}
\usepackage[catalan]{babel}
\usepackage{titlesec}
\usepackage{enumitem}
\usepackage{physics}
\usepackage{fancyhdr}

\newcommand{\NN}{\ensuremath{\mathbb{N}}} % set of natural numbers
\newcommand{\ZZ}{\ensuremath{\mathbb{Z}}} % set of integers
\newcommand{\QQ}{\ensuremath{\mathbb{Q}}} % set of rationals
\newcommand{\RR}{\ensuremath{\mathbb{R}}} % set of real numbers
\newcommand{\CC}{\ensuremath{\mathbb{C}}} % set of complex numbers
\newcommand{\KK}{\ensuremath{\mathbb{K}}} % a general field

\newcommand{\vf}[1]{\boldsymbol{\mathrm{#1}}} % math style for vectors and matrices and vector-values functions (previously it was \*vb{#1} but this does not apply to greek letters)
\newcommand{\ii}{\mathrm{i}} % imaginary unit
\renewcommand{\O}{\mathrm{O}} % big O-notation
\DeclareMathOperator{\supp}{\mathrm{supp}} % support
\allowdisplaybreaks

\newtheorem{theorem}{Teorema}
\newtheorem{exercici}{Exercici}
\newtheorem{prop}{Proposició}
\theoremstyle{definition}
\newtheorem{definition}{Definició}
\theoremstyle{remark}
\newtheorem*{res}{Resolució}
\DeclareDocumentCommand\derivative{ s o m g d() }{ 
  % Total derivative
  % s: star for \flatfrac flat derivative
  % o: optional n for nth derivative
  % m: mandatory (x in df/dx)
  % g: optional (f in df/dx)
  % d: long-form d/dx(...)
    \IfBooleanTF{#1}
    {\let\fractype\flatfrac}
    {\let\fractype\frac}
    \IfNoValueTF{#4}
    {
        \IfNoValueTF{#5}
        {\fractype{\diffd \IfNoValueTF{#2}{}{^{#2}}}{\diffd #3\IfNoValueTF{#2}{}{^{#2}}}}
        {\fractype{\diffd \IfNoValueTF{#2}{}{^{#2}}}{\diffd #3\IfNoValueTF{#2}{}{^{#2}}} \argopen(#5\argclose)}
    }
    {\fractype{\diffd \IfNoValueTF{#2}{}{^{#2}} #3}{\diffd #4\IfNoValueTF{#2}{}{^{#2}}}\IfValueT{#5}{(#5)}}
} % differential operator
\DeclareDocumentCommand\partialderivative{ s o m g d() }{ 
  % Total derivative
  % s: star for \flatfrac flat derivative
  % o: optional n for nth derivative
  % m: mandatory (x in df/dx)
  % g: optional (f in df/dx)
  % d: long-form d/dx(...)
  \IfBooleanTF{#1}
    {\let\fractype\flatfrac}
    {\let\fractype\frac}
    \IfNoValueTF{#4}{
      \IfNoValueTF{#5}
      {\fractype{\partial \IfNoValueTF{#2}{}{^{#2}}}{\partial #3\IfNoValueTF{#2}{}{^{#2}}}}
      {\fractype{\partial \IfNoValueTF{#2}{}{^{#2}}}{\partial #3\IfNoValueTF{#2}{}{^{#2}}} \argopen(#5\argclose)}
    }
    {\fractype{\partial \IfNoValueTF{#2}{}{^{#2}} #3}{\partial #4\IfNoValueTF{#2}{}{^{#2}}}\IfValueT{#5}{(#5)}}
} % partial differential operator

\titleformat{\section}
  {\normalfont\fontsize{11}{15}\bfseries}{\thesection}{1em}{}


%%% HARMONIC ANALYSIS
\newcommand{\F}{\mathcal{F}} % Fourier transform operator
\renewcommand{\pv}{\mathrm{p.v.}} % Cauchy principal value

% \renewcommand{\theenumi}{\textbf{\arabic{enumi}}}
\renewcommand{\theenumi}{\alph{enumi}}
\renewcommand{\theenumiii}{\roman{enumiii}}

\renewcommand{\exp}[1]{\mathrm{e}^{#1}} % exponential function
\DeclareMathOperator*{\im}{Im}

\title{\bfseries\Large Llista de problemes 4}

\author{Víctor Ballester Ribó\\NIU: 1570866}
\date{\parbox{\linewidth}{\centering
  Anàlisi Harmònica\endgraf
  Grau en Matemàtiques\endgraf
  Universitat Autònoma de Barcelona\endgraf
  Abril de 2023}}
  \pagestyle{fancy}
  \fancyhf{}
  \fancyhfoffset[L]{1cm}
  \fancyhfoffset[R]{1cm}
  \rhead{NIU: 1570866}
  \lhead{Víctor Ballester}
  \cfoot{\thepage}
  %\setlength{\headheight}{13.6pt}

\setlength{\parindent}{0pt}
\begin{document}
\selectlanguage{catalan}
\maketitle
\begin{exercici}
  Són distribucions $T_1$, $T_2$ i $T_3$ definides com $T_1=\sum_{k=1}^\infty \frac{\delta_{1/k}}{k^2}$, $T_2=\sum_{k=1}^\infty \frac{\delta_{1/k}}{k}$ i $T_3$ tal que $\langle T_3,\varphi\rangle=\sum_{n=1}^\infty\int_0^n \varphi'(x)\dd{x}$ per $\varphi\in\mathcal{D}(\RR)$?
\end{exercici}
\begin{res}
  Sigui $\varphi,\psi\in\mathcal{D}(\Omega)$. Tenim que
  $$T_1(\varphi+\psi)=\sum_{k=1}^\infty \frac{\delta_{1/k}(\varphi+\psi)}{k^2}=\sum_{k=1}^\infty \frac{\varphi(1/k)+\psi(1/k)}{k^2}=\sum_{k=1}^\infty \frac{\varphi(1/k)}{k^2}+\sum_{k=1}^\infty \frac{\psi(1/k)}{k^2}=T_1(\varphi)+T_1(\psi)$$
  on hem pogut reordenar perquè la sèrie absolutament convergent pel criteri M-Weierstra\ss\ ja que $\abs{\varphi(1/k)+\psi(1/k)}\leq \norm{\varphi}_\infty+\norm{\psi}_\infty=C\in\RR$. Per tant, $T_1$ és lineal. A més:
  $$
    \abs{T_1(\varphi)}\leq \sum_{k=1}^\infty \frac{\abs{\varphi(1/k)}}{k^2}\leq \frac{\pi^2}{6}\norm{\varphi}_\infty
  $$
  Per tant, $T_1$ és continu, per una proposició vista a classe.

  Ara veurem que tant $T_2$ com $T_3$ no són distribucions. Sigui
  $$
    \varphi(x)=\begin{cases}
      \exp{-\frac{1}{1-x^2}} & \text{si } x\in(-1,1) \\
      0                      & \text{altrament}
    \end{cases}
  $$
  Sabem que $\varphi\in\mathcal{D}(\RR)$. A més, $\varphi(0)=\exp{-1}$ i $\varphi(1/k)\geq \varphi(1/2) = \exp{-4/3}$ per $k\geq 2$. Per tant:
  \begin{align*}
    T_2(\varphi) & =\sum_{k=1}^{\infty}\frac{\delta_{1/k}(\varphi)}{k}=\sum_{k=1}^{\infty}\frac{\varphi(1/k)}{k}\geq\exp{-4/3}\sum_{k=2}^{\infty}\frac{1}{k}=\infty \\
    T_3(\varphi) & =\sum_{n=1}^{\infty}\int_0^n \varphi'(x)\dd{x}=\sum_{n=1}^{\infty}[\varphi(n)-\varphi(0)]=-\sum_{n=1}^{\infty}\exp{-1}=-\infty
  \end{align*}
  Per tant, hem trobat un element de $\mathcal{D}(\RR)$ que aplicat a $T_2$ i $T_3$ dona un resultat infinit (que no pertany a $\CC$). Per tant, no poden ser distribucions.
\end{res}
\begin{exercici}
  Considereu la distribució $T=\log\abs{x}$. Demostreu que $T'=\pv\frac{1}{x}$. Calculeu $T''$.
\end{exercici}
\begin{res}
  Recordem primer que $\log\abs{x}$ és localment integrable i que per tant té sentit considerar la seva distribució. Sigui $\varphi\in\mathcal{D}(\Omega)$. Tenim que:
  \begin{align*}
    \langle T',\varphi\rangle & = -\langle T, \varphi'\rangle                                                                              \\
                              & = -\int_{-\infty}^{+\infty} \log\abs{x}\varphi'(x)\dd{x}                                                   \\
                              & = -\int_{0}^{+\infty} \log(x)(\varphi'(x)+\varphi'(-x))\dd{x}                                              \\
                              & = -\int_{0}^{+\infty} \log(x){(\varphi(x)-\varphi(-x))}'\dd{x}                                             \\
                              & = -\log(x)(\varphi(x)-\varphi(-x))\Big|_0^\infty+\int_{0}^{+\infty} \frac{\varphi(x)-\varphi(-x)}{x}\dd{x} \\
                              & = \int_{0}^{+\infty} \frac{\varphi(x)-\varphi(-x)}{x}\dd{x}                                                \\
                              & =\left\langle\pv\left(\frac{1}{x}\right),\varphi\right\rangle
  \end{align*}
  on en la penúltima igualtat hem usat que $\varphi$ té suport compacte i que $\log(x)(\varphi(x)-\varphi(-x))\sim Cx\log(x)\to 0$ per $x\sim 0$. Calculem ara $T''={\left(\pv\left(\frac{1}{x}\right)\right)}'$:
  \begin{align*}
    \left\langle {\left(\pv\left(\frac{1}{x}\right)\right)}',\varphi\right\rangle & = -\left\langle \pv\left(\frac{1}{x}\right),\varphi'\right\rangle                                                                                                                                                                       \\
                                                                                  & =-\int_0^\infty \frac{\varphi'(x)-\varphi'(-x)}{x}\dd{x}                                                                                                                                                                                \\
                                                                                  & =-\int_0^\infty \frac{{\left[\varphi(x)+\varphi(-x)\right]}'}{x}\dd{x}                                                                                                                                                                  \\
                                                                                  & =-\frac{\varphi(x)+\varphi(-x)}{x}\bigg|_0^\infty-\int_0^\infty \frac{\varphi(x)+\varphi(-x)}{x^2}\dd{x}                                                                                                                                \\
                                                                                  & =\lim_{\varepsilon\to 0}\frac{\varphi(\varepsilon)+\varphi(-\varepsilon)}{\varepsilon}-\lim_{\varepsilon\to 0}\int_\varepsilon^\infty \left(\frac{\varphi(x)+\varphi(-x)-2\varphi(0)}{x^2}+\frac{2\varphi(0)}{x^2}\right)\dd{x}         \\
                                                                                  & =\lim_{\varepsilon\to 0}\left(\frac{\varphi(\varepsilon)+\varphi(-\varepsilon)}{\varepsilon}-\frac{2\varphi(0)}{\varepsilon}\right)-\lim_{\varepsilon\to 0}\int_\varepsilon^\infty \frac{\varphi(x)+\varphi(-x)-2\varphi(0)}{x^2}\dd{x} \\
                                                                                  & =-\int_0^\infty \frac{\varphi(x)+\varphi(-x)-2\varphi(0)}{x^2}\dd{x}
  \end{align*}
  on en l'última igualtat hem fet servir que $\varphi(\varepsilon)+\varphi(-\varepsilon)-2\varphi(0)\sim C\varepsilon^2$ per $\varepsilon\sim 0$ (surt fent Taylor). Per tant, la integral està ben definida al 0 (i també a l'infinit ja que podem acotar l'integrant per $\frac{4\norm{\varphi}_\infty}{x^2}$). A més com que és lineal (per la linealitat de la integral), tenim que $T''$ és una distribució que ve donada per:
  $$
    \varphi\longmapsto -\int_0^\infty \frac{\varphi(x)+\varphi(-x)-2\varphi(0)}{x^2}\dd{x}
  $$
\end{res}
\begin{exercici}
  Sigui $T\in \mathcal{D}^*(\RR)$ tal que $T'=0$. Demostreu que existeix una constant $C$ tal que $\langle T,\varphi\rangle=C\int_{-\infty}^{\infty} \varphi(t)\dd{t}$ $\forall \varphi\in\mathcal{D}(\RR)$.
\end{exercici}
\begin{res}
  Sigui $\varphi\in \mathcal{D}(\RR)$ tal que $\int_{-\infty}^{+\infty} \varphi(t)\dd{t}=0$. Tenim que si
  $$
    \psi(x):=\int_{-\infty}^x \varphi(t)\dd{t}
  $$
  aleshores $\psi$ té suport compacte, perquè $\varphi$ té suport compacte i integra 0, i a més $\psi$ és diferenciable i $\psi'(x)=\varphi(x)$ pel teorema fonamental del càlcul, i de fet és $\mathcal{C}^\infty(\RR)$. Per tant, $\psi\in\mathcal{D}(\RR)$. Llavors:
  $$
    T(\varphi)=T(\psi')=-T'(\psi)=0
  $$
  ja que $T'=0$ per hipòtesi. Per tant, $T(\varphi)=0$ $\forall \varphi\in\mathcal{D}(\RR)$ tal que $\int_{-\infty}^{+\infty} \varphi(t)\dd{t}=0$. Ara prenem $\phi\in \mathcal{D}(\RR)$ tal que $\int_{-\infty}^{+\infty} \phi(t)\dd{t}=1$ i sigui $\omega\in \mathcal{D}(\RR)$ arbitrària. Tenim que
  $$
    \varphi:=\omega-\phi\int_{-\infty}^{+\infty} \omega(t)\dd{t}
  $$
  és una funció de $\mathcal{D}(\Omega)$ que integra 0. Per tant, pel que hem vist anteriorment i la linealitat de $T$ obtenim:
  $$
    0=T(\varphi)=T\left(\omega-\phi\int_{-\infty}^{+\infty} \omega(t)\dd{t}\right)=T(\omega)-T(\phi)\int_{-\infty}^{+\infty} \omega(t)\dd{t}
  $$
  Per tant, la constant $C$ és $C:= T(\phi)$.
\end{res}
\begin{exercici}\hfill
  \begin{enumerate}
    \item Calculeu la derivada distribucional de la funció $f(x)=[x]$.
    \item Calculeu la derivada distribucional de la funció $$
            g(x)=\begin{cases}
              \frac{1}{\sqrt{x}} & \text{si } x>0     \\
              0                  & \text{si } x\leq 0 \\
            \end{cases}
          $$
  \end{enumerate}
\end{exercici}
\begin{res}\hfill
  \begin{enumerate}
    \item Assumim que $[x]=\lfloor x\rfloor$. Tenim que per $\varphi\in\mathcal{D}(\RR)$:
          $$
            \langle \lfloor x\rfloor',\varphi\rangle=-\langle\lfloor x\rfloor,\varphi'\rangle=-\int_{-\infty}^{+\infty} \lfloor x\rfloor\varphi'(x)\dd{x}=-\sum_{n\in\ZZ}n\int_{n}^{n+1} \varphi'(x)\dd{x}=-\sum_{n\in\ZZ}n[\varphi(n+1)-\varphi(n)]
          $$
          Si suposem que $\supp\varphi\subseteq [-N,N]$, tenim que:
          \begin{align*}
            \langle \lfloor x\rfloor',\varphi\rangle & =-\sum_{n=-N-1}^{N}n[\varphi(n+1)-\varphi(n)]                                                   \\
                                                     & =-\left[-(-N-1)\varphi(-N-1)+\sum_{n=-N}^{N}[(n-1)\varphi(n)-n\varphi(n)]+ N\varphi(N+1)\right] \\
                                                     & =\sum_{n=-N}^{N}\varphi(n)                                                                      \\
                                                     & =\sum_{n\in\ZZ} \delta_n(\varphi)
          \end{align*}
          Per tant, $\lfloor x\rfloor'=\sum_{n\in\ZZ}\delta_n$, que és una distribució perquè la suma és sempre finita.
    \item Tenim que per $\varphi\in\mathcal{D}(\RR)$:
          \begin{align*}
            \langle g',\varphi\rangle=-\langle g,\varphi'\rangle & =-\int_0^{\infty}\frac{\varphi'(x)}{\sqrt{x}}\dd{x}                                                                                                                                                                             \\
                                                                 & =-\frac{\varphi(x)}{\sqrt{x}}\bigg|_0^\infty-\frac{1}{2}\int_0^\infty \frac{\varphi(x)}{x^{3/2}}\dd{x}                                                                                                                          \\
                                                                 & =\lim_{\varepsilon\to 0}\frac{\varphi(\varepsilon)}{\sqrt{\varepsilon}}-\frac{1}{2}\lim_{\varepsilon\to 0}\int_\varepsilon^\infty \left(\frac{\varphi(x)-\varphi(0)}{x^{3/2}}+\frac{\varphi(0)}{x^{3/2}}\right)\dd{x}           \\
                                                                 & =\lim_{\varepsilon\to 0}\left(\frac{\varphi(\varepsilon)}{\sqrt{\varepsilon}}-\frac{\varphi(0)}{\sqrt{\varepsilon}}\right)-\frac{1}{2}\lim_{\varepsilon\to 0}\int_\varepsilon^\infty\frac{\varphi(x)-\varphi(0)}{x^{3/2}}\dd{x} \\
                                                                 & =-\frac{1}{2}\int_0^\infty\frac{\varphi(x)-\varphi(0)}{x^{3/2}}\dd{x}
          \end{align*}
          Per tant, com que la integral està ben definida en un entorn del 0 (ja que és de la forma $\frac{1}{\sqrt{x}}$) i com que també podem acotar l'integrant per $\frac{2\norm{\varphi}_\infty}{x^{3/2}}$, tenim que també està ben definida a l'infinit i per tant és distribució (ja que també és lineal, per la linealitat del la integral). Així doncs, $g'$ és la distribució donada per:
          $$
            \varphi\longmapsto -\frac{1}{2}\int_0^\infty\frac{\varphi(x)-\varphi(0)}{x^{3/2}}\dd{x}
          $$
  \end{enumerate}
\end{res}
\end{document}
