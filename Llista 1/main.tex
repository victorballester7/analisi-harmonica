\documentclass[10pt,a4paper]{article}
\usepackage[utf8]{inputenc}
\usepackage{amsthm, amsmath, mathtools, amssymb}
\usepackage[left=2cm,right=2cm,top=2cm,bottom=2cm]{geometry}
\usepackage[colorlinks,linkcolor=blue,citecolor=blue,urlcolor=blue]{hyperref}
\usepackage[catalan]{babel}
\usepackage{titlesec}
\usepackage{enumitem}
\usepackage{physics}
\usepackage{fancyhdr}

\newcommand{\NN}{\ensuremath{\mathbb{N}}} % set of natural numbers
\newcommand{\ZZ}{\ensuremath{\mathbb{Z}}} % set of integers
\newcommand{\QQ}{\ensuremath{\mathbb{Q}}} % set of rationals
\newcommand{\RR}{\ensuremath{\mathbb{R}}} % set of real numbers
\newcommand{\CC}{\ensuremath{\mathbb{C}}} % set of complex numbers
\newcommand{\KK}{\ensuremath{\mathbb{K}}} % a general field

\newcommand{\vf}[1]{\boldsymbol{\mathrm{#1}}} % math style for vectors and matrices and vector-values functions (previously it was \*vb{#1} but this does not apply to greek letters)
\newcommand{\ii}{\mathrm{i}} % imaginary unit
\renewcommand{\O}{\mathrm{O}} % big O-notation

\newtheorem{theorem}{Teorema}
\newtheorem{exercici}{Exercici}
\newtheorem{prop}{Proposició}
\theoremstyle{definition}
\newtheorem{definition}{Definició}
\theoremstyle{remark}
\newtheorem*{res}{Resolució}
\DeclareDocumentCommand\derivative{ s o m g d() }{ 
  % Total derivative
  % s: star for \flatfrac flat derivative
  % o: optional n for nth derivative
  % m: mandatory (x in df/dx)
  % g: optional (f in df/dx)
  % d: long-form d/dx(...)
    \IfBooleanTF{#1}
    {\let\fractype\flatfrac}
    {\let\fractype\frac}
    \IfNoValueTF{#4}
    {
        \IfNoValueTF{#5}
        {\fractype{\diffd \IfNoValueTF{#2}{}{^{#2}}}{\diffd #3\IfNoValueTF{#2}{}{^{#2}}}}
        {\fractype{\diffd \IfNoValueTF{#2}{}{^{#2}}}{\diffd #3\IfNoValueTF{#2}{}{^{#2}}} \argopen(#5\argclose)}
    }
    {\fractype{\diffd \IfNoValueTF{#2}{}{^{#2}} #3}{\diffd #4\IfNoValueTF{#2}{}{^{#2}}}\IfValueT{#5}{(#5)}}
} % differential operator
\DeclareDocumentCommand\partialderivative{ s o m g d() }{ 
  % Total derivative
  % s: star for \flatfrac flat derivative
  % o: optional n for nth derivative
  % m: mandatory (x in df/dx)
  % g: optional (f in df/dx)
  % d: long-form d/dx(...)
  \IfBooleanTF{#1}
    {\let\fractype\flatfrac}
    {\let\fractype\frac}
    \IfNoValueTF{#4}{
      \IfNoValueTF{#5}
      {\fractype{\partial \IfNoValueTF{#2}{}{^{#2}}}{\partial #3\IfNoValueTF{#2}{}{^{#2}}}}
      {\fractype{\partial \IfNoValueTF{#2}{}{^{#2}}}{\partial #3\IfNoValueTF{#2}{}{^{#2}}} \argopen(#5\argclose)}
    }
    {\fractype{\partial \IfNoValueTF{#2}{}{^{#2}} #3}{\partial #4\IfNoValueTF{#2}{}{^{#2}}}\IfValueT{#5}{(#5)}}
} % partial differential operator

\titleformat{\section}
  {\normalfont\fontsize{11}{15}\bfseries}{\thesection}{1em}{}

% \renewcommand{\theenumi}{\textbf{\arabic{enumi}}}
\renewcommand{\theenumi}{\alph{enumi}}
\renewcommand{\theenumiii}{\roman{enumiii}}
\renewcommand{\exp}[1]{\mathrm{e}^{#1}} % exponential function
\DeclareMathOperator*{\im}{Im}

\title{\bfseries\Large Llista de problemes 1}

\author{Víctor Ballester Ribó\\NIU: 1570866}
\date{\parbox{\linewidth}{\centering
  Anàlisi Harmònica\endgraf
  Grau en Matemàtiques\endgraf
  Universitat Autònoma de Barcelona\endgraf
  Febrer de 2023}}
  \pagestyle{fancy}
  \fancyhf{}
  \fancyhfoffset[L]{1cm}
  \fancyhfoffset[R]{1cm}
  \rhead{NIU: 1570866}
  \lhead{Víctor Ballester}
  \cfoot{\thepage}
  %\setlength{\headheight}{13.6pt}

\setlength{\parindent}{0pt}
\begin{document}
\selectlanguage{catalan}
\maketitle
\begin{exercici}
  Escriviu les sèries de Fourier de sinus i de cosinus de la funció
  \begin{equation*}
    f(x)=\begin{cases}
      \frac{\pi}{3}  & \text{si } x\in(0,\pi/3)      \\
      0              & \text{si } x\in(\pi/3,2\pi/3) \\
      -\frac{\pi}{3} & \text{si } x\in(2\pi/3,\pi)
    \end{cases}
  \end{equation*}
\end{exercici}
\begin{res}
  Per calcular la del sinus, fem l'extensió senar de $f$ en $[-\pi,\pi]$ i llavors sabem que la seva sèrie de Fourier és $$S_1f(x)=\sum_{k=1}^{\infty}b_k\sin(kx)$$ on $b_k=\frac{2}{\pi}\int_0^\pi f(x)\sin(kx)$. Tenim que:

  \begin{align*}
    b_k & =\frac{2}{\pi}\int_0^{\frac{\pi}{3}} \frac{\pi}{3}\sin(kx)-\frac{2}{\pi}\int_{\frac{2\pi}{3}}^{\pi} \frac{\pi}{3}\sin(kx) \\
        & =\frac{2}{3}\frac{1+{(-1)}^k-\cos(k\pi/3)-\cos(2k\pi/3)}{k}                                                               \\
        & =\frac{2}{3}\frac{1+{(-1)}^k-2\cos(k\pi/2)\cos(k\pi/6)}{k}                                                                \\
        & =\begin{cases}
             0                                & \text{si $k$ és senar}  \\
             0                                & \text{si $k=6\ell$}     \\
             \displaystyle\frac{1}{3\ell + 1} & \text{si $k=6\ell + 2$} \\
             \displaystyle\frac{1}{3\ell + 2} & \text{si $k=6\ell + 4$}
           \end{cases}
  \end{align*}
  on hem usat la identitat trigonomètrica $\cos a+\cos b=2\cos\left(\frac{a+b}{2}\right)\cos\left(\frac{a-b}{2}\right)$. Així doncs: $$S_1f(x)=\sum_{\ell = 0}^\infty\left(\frac{\sin((6\ell + 2)x)}{3\ell + 1} + \frac{\sin((6\ell + 4)x)}{3\ell + 2} \right)$$

  Si ara fem l'extensió parell de $f$ en $[-\pi,\pi]$ obtenim la seva sèrie de Fourier en termes del cosinus:
  \begin{equation}\label{cosinus}
    S_2f(x)=\frac{a_0}{2}+\sum_{k=1}^{\infty}a_k\cos(kx)
  \end{equation}
  on $a_k=\frac{2}{\pi}\int_0^\pi f(x)\cos(kx)$.
  Tenim que:

  \begin{align*}
    a_k & =\frac{2}{\pi}\int_0^{\frac{\pi}{3}} \frac{\pi}{3}\cos(kx)-\frac{2}{\pi}\int_{\frac{2\pi}{3}}^{\pi} \frac{\pi}{3}\cos(kx) \\
        & =\begin{cases}
             \frac{2}{3}\frac{\sin(k\pi/3)+\sin(2k\pi/3)}{k} & \text{si $k\ne 0$} \\
             0                                               & \text{si $k=0$}
           \end{cases}                                                     \\
        & =\begin{cases}
             \frac{4}{3}\frac{\sin(k\pi/2)\cos(k\pi/6)}{k} & \text{si $k\ne 0$} \\
             0                                             & \text{si $k=0$}
           \end{cases}                                                       \\
        & =\begin{cases}
             0                                          & \text{si $k$ és parell} \\
             \displaystyle\frac{2\sqrt{3}}{3(6\ell +1)} & \text{si $k=6\ell+1$}   \\
             0                                          & \text{si $k=6\ell + 3$} \\
             \displaystyle-\frac{2\sqrt{3}}{3(6\ell+5)} & \text{si $k=6\ell + 5$}
           \end{cases}
  \end{align*}
  on hem usat la identitat trigonomètrica $\sin a+\sin b=2\sin\left(\frac{a+b}{2}\right)\cos\left(\frac{a-b}{2}\right)$.
  Així doncs:
  $$S_2f(x)=\frac{2\sqrt{3}}{3}\sum_{\ell = 0}^\infty\left(\frac{\cos((6\ell + 1)x)}{6\ell + 1} - \frac{\cos((6\ell + 5)x)}{6\ell + 5}\right)$$
\end{res}
\begin{exercici}
  Sigui $f \in\mathcal{C}^k$ una funció $2\pi$-periòdica. Demostreu que $\widehat{f}(n)=\O(\abs{n}^{-k})$ quan $\abs{n}\to\infty$ (i.e. existeix una constant $C > 0$ tal que $\abs{\widehat{f}(n)}\leq C\abs{n}^{-k}$).
\end{exercici}
\begin{res}
  Demostrem per inducció sobre $k$ que $$\widehat{f^{(k)}}(n)={(\ii n)}^k\widehat{f}(n)$$
  El cas $k=0$ és directe. Si suposem cert el cas $k-1$ tenim que:
  \begin{align*}
    \widehat{f^{(k)}}(n) & =\left\langle f^{(k)}(x),\frac{1}{2\pi}\exp{\ii nx}\right\rangle                                                               \\
                         & =\frac{1}{2\pi}\int_{-\pi}^\pi f^{(k)}(x)\exp{-\ii n x}\dd{x}                                                                  \\
                         & =\frac{1}{2\pi}f^{(k-1)}(x)\exp{-\ii n x}\Big|_{-\pi}^\pi + \frac{\ii n}{2\pi}\int_{-\pi}^\pi f^{(k-1)}(x)\exp{-\ii n x}\dd{x} \\
                         & =\ii n \widehat{f^{(k-1)}}(n)                                                                                                  \\
                         & ={(\ii n)}^k\widehat{f}(n)
  \end{align*}
  on hem fet integració per parts i hem fet servir la continuïtat de $f^{(k-1)}$ en els punts ``d'unió"\ ($f^{(k-1)}(-\pi)=f^{(k-1)}(\pi)$) i la hipòtesi d'inducció en l'última igualtat.
  Per tant, com que sempre tenim que $$\abs{\widehat{g}(n)}\leq\frac{1}{2\pi}\norm{g}_1$$ i $\norm{f^{(k)}}_1=C<\infty$ perquè $f^{(k)}$ és contínua, deduïm que:
  $$\abs{\widehat{f}(n)}=\abs{\frac{\widehat{f^{(k)}}(n)}{{(\ii n)}^k}}\leq C\abs{n}^{-k}$$
\end{res}
\begin{exercici}
  A l'interval $[-\pi, \pi]$ considereu la funció
  \begin{equation*}
    f(t)=\begin{cases}

      0                        & \text{si } \abs{t}>\delta    \\
      1-\frac{\abs{t}}{\delta} & \text{si } \abs{t}\leq\delta
    \end{cases}
  \end{equation*}
  Demostreu que $$f(t)=\frac{\delta}{2\pi}+2\sum_{n=1}^\infty\frac{1-\cos(n\delta)}{n^2\pi\delta}\cos(nt)$$
\end{exercici}
\begin{res}
  La funció $f(t)$ és parella, per tant la seva sèrie de Fourier només tindrà els termes del cosinus. Aquests són els següents:
  \begin{align*}
    a_n & =\frac{2}{\pi}\int_0^\pi f(t)\cos(nt)\dd{t}                                \\
        & =\frac{2}{\pi}\int_0^\delta \left(1-\frac{t}{\delta}\right)\cos(nt)\dd{t}  \\
        & =\begin{cases}
             \displaystyle 2\frac{1-\cos(n\delta)}{\pi n^2\delta} & \text{si $n\ne 0$} \\
             \frac{\delta}{\pi}                                   & \text{si $n=0$}
           \end{cases}
  \end{align*}
  D'aquí (i recordant l'expressió \eqref{cosinus}) es desprèn automàticament el resultat ja que $f$ és contínua i derivable excepte a un nombre finit de punts i la derivada està acotada. De fet, per aquest motiu la convergència de $Sf$ a $f$ és uniforme.
\end{res}
\begin{exercici}
  Proveu que els coeficients de Fourier es poden escriure com $$a_k=\frac{1}{2\pi}\int_{-\pi}^\pi\left[f(x)-f\left(x-\frac{\pi}{k}\right)\right]\cos(kx)\dd{x}\qquad b_k=\frac{1}{2\pi}\int_{-\pi}^\pi\left[f(x)-f\left(x-\frac{\pi}{k}\right)\right]\sin(kx)\dd{x}$$
  Deduïu que si $f$ satisfà una condició Hölder d'ordre $\alpha$, i.e. $\abs{f(x)-f(y)}\leq L\abs{x-y}^\alpha$, llavors els coeficients de Fourier satisfan $$\abs{a_k}\leq L\frac{\pi^\alpha}{k^\alpha}\qquad\abs{b_k}\leq L\frac{\pi^\alpha}{k^\alpha}$$
\end{exercici}
\begin{res}
  Per demostrar les igualtats és suficient veure que $$a_k=-\frac{1}{\pi}\int_{-\pi}^\pi f\left(x-\frac{\pi}{k}\right)\cos(kx)\dd{x}\qquad b_k=-\frac{1}{\pi}\int_{-\pi}^\pi f\left(x-\frac{\pi}{k}\right)\sin(kx)\dd{x}$$
  Fent el canvi $y=x-\frac{\pi}{k}$ tenim que:
  \begin{align*}
    -\frac{1}{\pi}\int_{-\pi}^\pi f\left(x-\frac{\pi}{k}\right)\cos(kx)\dd{x} & = -\frac{1}{\pi}\int_{-\pi-\frac{\pi}{k}}^{\pi-\frac{\pi}{k}} f(y)\cos(ky+\pi)\dd{y} \\
                                                                              & =\frac{1}{\pi}\int_{-\pi-\frac{\pi}{k}}^{\pi-\frac{\pi}{k}} f(y)\cos(ky)\dd{y}       \\
                                                                              & =\frac{1}{\pi}\int_{-\pi}^{\pi} f(y)\cos(ky)\dd{y}                                   \\
                                                                              & =a_k
  \end{align*}
  La penúltima igualtat es deu al fet que si $g$ és $T$-periòdica i integrable, aleshores $\forall x\in\RR$ tenim $$\int_x^{x+T}g(x)\dd{x}=\int_0^{T}g(x)\dd{x}$$
  Similarment:
  \begin{align*}
    -\frac{1}{\pi}\int_{-\pi}^\pi f\left(x-\frac{\pi}{k}\right)\sin(kx)\dd{x} & = -\frac{1}{\pi}\int_{-\pi-\frac{\pi}{k}}^{\pi-\frac{\pi}{k}} f(y)\sin(ky+\pi)\dd{y} \\
                                                                              & =\frac{1}{\pi}\int_{-\pi-\frac{\pi}{k}}^{\pi-\frac{\pi}{k}} f(y)\sin(ky)\dd{y}       \\
                                                                              & =\frac{1}{\pi}\int_{-\pi}^{\pi} f(y)\sin(ky)\dd{y}                                   \\
                                                                              & =b_k
  \end{align*}
  A partir d'aquestes igualtats si $f$ satisfà la condició de Hölder tenim:
  \begin{align*}
    \abs{a_k} & \leq\frac{1}{2\pi}\int_{-\pi}^\pi\abs{\left[f(x)-f\left(x-\frac{x}{k}\right)\right]\cos(kx)}\dd{x}\leq \frac{L}{2\pi}\int_{-\pi}^\pi\frac{\abs{\pi}^\alpha}{k^\alpha}\dd{x}= L\frac{\pi^\alpha}{k^\alpha} \\
    \abs{b_k} & \leq\frac{1}{2\pi}\int_{-\pi}^\pi\abs{\left[f(x)-f\left(x-\frac{x}{k}\right)\right]\sin(kx)}\dd{x}\leq \frac{L}{2\pi}\int_{-\pi}^\pi\frac{\abs{\pi}^\alpha}{k^\alpha}\dd{x}=L\frac{\pi^\alpha}{k^\alpha}  \\
  \end{align*}
\end{res}
\end{document}
